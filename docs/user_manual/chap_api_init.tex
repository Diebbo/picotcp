\section{Initialization}

\subsection{pico$\_$stack$\_$init}

\subsubsection*{Description}
Initializes the picoTCP stack and must be called before any other function.

\subsubsection*{Function prototype}
\begin{verbatim}
int pico_stack_init(struct pico_stack **S);
\end{verbatim}

\subsubsection*{Parameters}
\begin{itemize}[noitemsep]
\item \texttt{S} - Pointer to a pointer to store a reference to the picoTCP stack context being created.
\end{itemize}

\subsubsection*{Return value}
On success, this call returns 0.
On error, -1 is returned (\texttt{pico$\_$err} is \textbf{not} set).

\subsubsection*{Example}
\begin{verbatim}
int main(int argc, char* argv[]) {
    struct pico_stack *S;
    pico_stack_init(&S);
    /* Code using picoTCP */
}
\end{verbatim}
